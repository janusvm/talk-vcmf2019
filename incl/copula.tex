\section{Recap: Copulas and Vines}
\label{sec:copula}

\begin{frame}{Recap on Copulas}{Definition}
  \begin{defn}[Copula]
    A \emph{\(d\)-dimensional copula} is a function \(C : \intcc{0,1}^{d} \to \intcc{0,1}\) such that
    \begin{enumerate}
    \item \(C\of{u_{1}, \dotsc, u_{d}} = 0\) if \(u_{j} = 0\) for at least one \(j \in \cbr{1,\dotsc,d}\)
    \item \(C\of{1, \dotsc, 1, u_{j}, 1, \dotsc, 1} = u_{j}\) for all \(j \in \cbr{1,\dotsc,d}\)
    \item \(V_{C}\of{\intoc{\va, \vb}} \geq 0\) for every \(d\)-box \(\intoc{\va,\vb} \subseteq \intcc{0,1}^{d}\)
    \end{enumerate}
    where \(\intoc{\va,\vb} \coloneqq \intoc{a_{1}, b_{1}} \times \dotsb \intoc{a_{d}, b_{d}}\) and
    \begin{align*}
      V_{C}\of{\intoc{\va,\vb}} &\coloneqq \sum_{\vv\in\ver\intoc{\va,\vb}}\sign\of{\vv} C\of{\vv} ; \\
      \ver\intoc{\va,\vb} &\coloneqq \cbr{a_{1}, b_{1}} \times \dotsb \times \cbr{a_{d}, b_{d}} , \\
      \sign\of{\vv} &\coloneqq
                      \begin{cases}
                        1, & \text{if}~ v_{j} = a_{j} ~\text{for an even number of indices}, \\
                        -1, & \text{otherwise} .
                      \end{cases}
    \end{align*}
  \end{defn}
\end{frame}

\begin{frame}{Recap on Copulas}{Sklar's Theorem}
  \begin{thm}[Sklar's Theorem]
    Let \(\vX\) be a \(d\)-dimensional random vector with joint distribution function \(F\) and marginals \(F_{1}, \dotsc, F_{d}\).
    Then there exists a \(d\)-copula \(C\) such that
    \begin{equation*}
      F\of{\vx} = C\of{F_{1}\of{x_{1}}, \dotsc, F_{d}\of{x_{d}}}, \quad \forall \vx \in \R^{d}
    \end{equation*}
    and if the marginals are all continous, then \(C\) is uniquely defined.
  \end{thm}
\end{frame}

\begin{frame}{Recap on Copulas}{Families of Copulas \((d=2)\)}
  \begin{center}
    \includegraphics[width=0.85\textwidth]{img/gen/copfamilies}
  \end{center}
\end{frame}

\begin{frame}{Recap on Copulas}{Basic Idea Behind Vine Copulas}
  \begin{itemize}
  \item<2-> Factor full joint density into product of marginal densities and copula densities
  \item<3-> Represent with sequence of trees, such that
    \begin{itemize}
    \item Marginals are nodes in the first tree
    \item Pair-copulas between the marginals are edges in the first tree
    \item Edges from the previous tree becomes nodes in the next
    \item Edges in subsequent trees are conditional pair-copulas
    \item Nodes in a tree can be joined by an edge, if the nodes share a node from the previous tree
    \end{itemize}
  \item<4-> The chosen pair-copulas need not be the same
  \end{itemize}
\end{frame}

\begin{frame}{Recap on Copulas}{Basic Idea Behind Vine Copulas: Example}
  \begin{block}{Example in \(d = 4\)}
    \begin{center}
      \includegraphics[width=0.8\textwidth]{img/tikz/vine4.tikz}
    \end{center}
  \end{block}
\end{frame}

%%% Local Variables:
%%% mode: latex
%%% TeX-master: "../slides"
%%% End:
